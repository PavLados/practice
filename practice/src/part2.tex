% ---------------------------- Problem 1----------------------------------
\subsubsection*{\center Задача № 1.}
{\bf Условие.~}
Дана последовательность $\{a_n\} = \dfrac{3n^3}{n^3-1}$ и число $c=3$. Доказать, что 
$$\lim_{n\to\infty}a_n=c,$$
а именно, для каждого сколь угодно малого числа $\eps>0$ найти наименьшее натуральное число 
$N=N(\eps)$ такое, что $|a_n-c|<\eps$ для всех номеров $n>N(\eps)$.
Заполнить таблицу
\begin{center}
	\begin{tabular}{|c|c|c|c|}
		\hline
		$\eps$ &  $0{,}1$ & $0{,}01$ & $0{,}001$ \\
		\hline
		$N(\eps)$ & & & \\
		\hline
	\end{tabular}
\end{center}
{\bf Решение.~}	
Рассмотрим неравенство $a_n-c<\eps,\,\forall\eps>0$, учитывая выражение для $a_n$ и значение $c$ из условия варианта,
получим
$$
\biggl|\frac{3n^3}{n^3-1}-3\biggr| < \eps.
$$
Неравенство запишем в виде двойного неравентсва и приведём выражение под знаком модуля к общему знаменателю,
получим
$$
-\eps < \frac{3}{n^3-1} < \eps.
$$
Заметим, что левое неравенство выполнено для любого номера $n\in\mathbb{N}$ поэтому, будем рассматривать правое неравенство 
$$
\frac{3}{n^3-1} < \eps.
$$
Выполнив цепочку преобразований, перепишем неравенство относительно $n^3$, и учитывая, что $n\in\mathbb{N}$, получим
$$
\begin{array}{c}
\dfrac{3}{n^3-1} < \eps, 							\\[8pt]
n^3-1 > \dfrac{3}{\eps}, 							\\[8pt]
n^3 > \dfrac{3}{\eps}+1, 	                        \\[8pt]
n > \sqrt[3]{\dfrac{3}{\eps}+1}, 		\\[8pt]
N(\eps) = \Biggl[\,\sqrt[3]{\dfrac{3}{\eps}+1}\,\Biggr],
\end{array}
$$
где $[\phantom{a}]$ --- целая часть числа.
Заполним таблицу:
\begin{center}
	\begin{tabular}{|c|c|c|c|}
		\hline
		$\eps$ &  $0{,}1$ & $0{,}01$ & $0{,}001$ \\
		\hline
		$N(\eps)$ & 3 & 6 & 14 \\
		\hline
	\end{tabular}
\end{center}
\textbf{Проверка:}
$$
\begin{array}{l}
|a_4 - c| = \dfrac{1}{121} < 0{,}1,			\\[10pt]
|a_7 - c| = \dfrac{1}{114} < 0{,}01,  	\\[10pt]
|a_{15} - c| = \dfrac{3}{3374} < 0{,}001.
\end{array}
$$

% ---------------------------- Problem 2----------------------------------
\subsubsection*{\center Задача № 2.}
{\bf Условие.~}
Вычислить пределы функций
$$
\begin{array}{cc}
\text{\bf(а):} & \lim\limits_{x\to\, -1}\dfrac{x^3+x^2-x-1}{x^3+5x^2+7x+3}, \\[15pt]
\text{\bf(б):} & \lim\limits_{x\to\infty}\dfrac{\sqrt{\,3x^4+2\sqrt[3]{x^{16}-4x}}}{\sqrt[3]{x^8+x^2-1}}, \\[15pt]
\text{\bf(в):} & \lim\limits_{x\to\,0}\dfrac{\sqrt[3]{1+x^2}-1}{\sqrt{1+x^2}-1}, \\[15pt]
\text{\bf(г):} & \lim\limits_{x\to\,0}(1-x\sin x )^{\dfrac{1}{\ln(1+\pi x^2)}}, \\[15pt]
\text{\bf(д):} & \lim\limits_{x\to\,0}(1-\arcsin x)^{\arcctg x}, \\[10pt]
\text{\bf(е):} & \lim\limits_{x\to\,1}\dfrac{x^2-1}{\ln x}.
\end{array}
$$
{\bf Решение.~}\\
\text{\bf(а):}
$$
\begin{array}{l}
\lim\limits_{x\to\,-1}\dfrac{x^3+x^2-x-1}{x^3+5x^2+7x+3} = 
\lim\limits_{x\to\,-1}\dfrac{(x+1)(x+1)(x-1)}{(x+1)(x+1)(x+3)} =
\lim\limits_{x\to\,-1}\dfrac{x-1}{x+3} =
\dfrac{-2}{2} = -1.
\end{array}
$$	
\text{\bf(б):}
$$
\begin{array}{1}
\lim\limits_{x\to\infty}\dfrac{\sqrt{\,3x^4+2\sqrt[3]{x^{16}-4x}}}{\sqrt[3]{x^8+x^2-1}} =
\lim\limits_{x\to\infty}\dfrac{\sqrt[3]{x^8}\sqrt{\frac{3}{\sqrt[3]{x^2}}+2\sqrt[3]{1-\frac{4}{x^{15}}}}}{\sqrt[3]{x^8}\sqrt[3]{1+\frac{1}{x^6}-\frac{1}{x^8}}} = \\
\lim\limits_{x\to\infty}\dfrac{\sqrt{\frac{3}{\sqrt[3]{x^2}}+2\sqrt[3]{1-\frac{4}{x^{15}}}}}{\sqrt[3]{1+\frac{1}{x^6}-\frac{1}{x^8}}} = \sqrt{2}.
\end{array}
$$
\text{\bf(в):}
$$
\begin{array}{1}
\lim\limits_{x\to\,0}(1-x\sin x )^{\frac{1}{\ln(1+\pi x^2)}}= e^{\;\lim\limits_{x\to\,0}(1-x\sin x-1)\bigl(\frac{1}{\ln(1+\pi x^2)}\bigr)}=\\
e^{\: -\frac 1\pi\lim\limits_{x\to\,0}\frac{\sin x}{x}} = e^{-\frac1\pi}.
\end{array}
$$
\text{\bf(д):}
$$
\begin{array}{1}
\lim\limits_{x\to\,0}(1-\arcsin x)^{\arcctg x} = (1-0)^0=1.
\end{array}
$$
\text{\bf(е):}
$$
\begin{array}{l}
 \lim\limits_{x\to\,1}\dfrac{x^2-1}{\ln x} = 
 \biggl|
 \begin{array}
\,t=x-1\\ t\to\,0
 \end {array}
 \biggr| = \lim\limits_{t\to\,0}\dfrac{(t+1-1)(t+1+1)}{t}=2.
\end{array}
$$


% ---------------------------- Problem 3----------------------------------
\subsubsection*{\center Задача № 3.}
{\bf Условие.~}\\
\text{\bf(а):} Показать, что данные функции
$f(x)$ и $g(x)$ являются бесконечно малыми или бесконечно большими
при указанном стремлении аргумента. \\
\text{\bf(б):} Для каждой функции $f(x)$ и $g(x)$ записать главную часть
(эквивалентную ей функцию)  вида $C(x-x_0)^{\alpha}$ при $x\rightarrow x_0$ или $Cx^{\alpha}$
при $x\rightarrow\infty$, указать их порядки малости (роста). \\
\text{\bf(в):} Сравнить функции $f(x)$ и $g(x)$ при указанном стремлении.
\begin{center}
	\begin{tabular}{|c|c|c|}
		\hline
		№ варианта & функции $f(x)$ и $g(x)$ & стремление \\[6pt]
		%\hline
		16 & $f(x) = \dfrac{2x^5+x+1}{x^4-3x^2+2},~g(x)=x^2\sin\dfrac{1}{\sqrt{x}}$ & $x\rightarrow+\infty$ \\ [10pt]
		\hline
	\end{tabular}
\end{center}
{\bf Решение.~}\\
\text{\bf(а):}~Покажем, что $f(x)$ и $g(x)$ бесконечно большие функции,
$$
\begin{array}{cc}
\lim\limits_{x\to+\infty}f(x) = \lim\limits_{x\to+\infty}\dfrac{2x^5+x+1}{x^4-3x^2+2} =
\lim\limits_{x\to+\infty}\dfrac{x^5\left(2+\frac{1}{x^4}+\frac{1}{x^5}\right)}{x^4\left(1-\frac{3}{x^2}+\frac{2}{x^4}\right)} =
\lim\limits_{x\to\,+\infty}2x=+\infty\\[20pt]
\lim\limits_{x\rightarrow+\infty}g(x) = \lim\limits_{x\rightarrow+\infty}x^2\sin\dfrac{1}{\sqrt{x}} =
\lim\limits_{x\rightarrow+\infty}x^2 \dfrac{1}{\sqrt{x}}= 
\lim\limits_{x\rightarrow+\infty}\sqrt{x^3}} = +\infty
\end{array}
$$	
\text{\bf(б):}~Так как $f(x)$ и $g(x)$ бесконечно большие функции, то эквивалентными им будут функции вида 
$Cx^{\alpha}$ при $x\rightarrow\+\infty$. Найдём эквивалентную для $f(x)$ из условия
$$
\lim\limits_{x\rightarrow\infty}\dfrac{f(x)}{x^{\alpha}} = C,
$$
где $C$ --- некоторая константа. Рассмотрим предел
$$
\lim\limits_{x\rightarrow+\infty}\dfrac{f(x)}{x^\alpha}=
\lim\limits_{x\to+\infty}\dfrac{2x^5+x+1}{x^\alpha (x^4-3x^2+2)} =
\lim\limits_{x\to+\infty}\dfrac{x^5\left(2+\frac{1}{x^4}+\frac{1}{x^5}\right)}{x^\alpha\cdot x^4\left(1-\frac{3}{x^2}+\frac{2}{x^4}\right)} =2
$$
При $\alpha=1$ последний предел равен $2$, отсюда $C=2$ и 
$$
f(x)\sim 2x~\text{при}~x\rightarrow+\infty.
$$
Аналогично, рассмотрим предел
$$
\lim\limits_{x\rightarrow\infty}\dfrac{g(x)}{x^{\alpha}} = 
\lim\limits_{x\rightarrow+\infty}\dfrac{x^2\sin\dfrac{1}{\sqrt{x}}}{x^\alpha} =
\lim\limits_{x\rightarrow+\infty}\dfrac{\sqrt {x^3}}{x^\alpha}=1
$$
При $\alpha=\dfrac 32$ последний предел равен $1$, отсюда $C=1$ и
$$
g(x)\sim x^{\frac 32}~\text{при}~x\rightarrow+\infty.
$$
\text{\bf(в):}~Для сравнения функций $f(x)$ и $g(x)$ рассмотрим предел их отношения при указанном стремлении
$$
\lim\limits_{x\rightarrow+\infty}\dfrac{f(x)}{g(x)}.
$$
Применим эквивалентности, определенные в пункте (б), получим
$$
\lim\limits_{x\rightarrow+\infty}\dfrac{f(x)}{g(x)} = 
\lim\limits_{x\rightarrow+\infty}\dfrac{2x}{x^\frac 32} = 
\lim\limits_{x\rightarrow+\infty} \dfrac{2}{\sqrt x} = 0.  
$$
Отсюда, $g(x)$ есть бесконечно большая функция более высокого порядка роста, чем $f(x)$.

